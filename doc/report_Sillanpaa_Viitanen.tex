\documentclass[a4paper]{article}
\usepackage[english]{babel}
\usepackage[utf8]{inputenc}
\usepackage[T1]{fontenc}
\usepackage{times}
\usepackage{float}
\usepackage{listings}
\usepackage[pdftex]{graphicx}
\usepackage{epstopdf}
\usepackage{pdfpages}
\usepackage{color}
\usepackage[pdftex,colorlinks=true,citecolor=black,
            pagecolor=black,linkcolor=black,menucolor=black,
            urlcolor=black]{hyperref}
\usepackage{eufrak}
\usepackage{amsmath}
\usepackage{amsbsy}
\usepackage{eucal}
%\usepackage{subfigure}
\usepackage{longtable}
\usepackage{url}
\urlstyle{same}

\usepackage{natbib}

\parindent 0mm
\parskip 3mm

\newcommand*\mean[1]{\bar{#1}}

\pdfinfo{            
          /Title      (T-61.5140 Machine Learning: Advanced Probabilistic Methods Project)
          /Author     (Ville Sillanpää, Lauri Viitanen)
          /Keywords   ()
}

\title{Classify the glass samples using Gaussian Mixture Models (GMM)}
\author{Ville Sillanpää, k84338 - Lauri Viitanen, 338853 \\ 
       {\it ville.sillanpaa@aalto.fi} -
       {\it lauri.viitanen@aalto.fi}}

\begin{document}
\maketitle
\clearpage

%\begin{figure}[]
%	\includegraphics[]{all_338853_2001.pdf}
%\end{figure}

\section{Introduction}

\begin{equation}
\begin{aligned}
	r[n] &= h[n] * s[n] \leftrightarrow
		R(e^{j\omega}) = H(e^{j\omega})S(e^{j\omega}) \\
	r[n] &= \sum_{k=-\infty}^{\infty} h[k]s[n-k] \\
\end{aligned}
\end{equation}

%Figure \ref{fig:100_104} shows combinations of these numbers

%\begin{figure}[H]
%	\includegraphics[scale=0.67]{m100_115_b.eps}
%	\caption{The amplitude response both in linear and logarithmic
%		(decibel) scale.}
%	\label{fig:100_115_b}
%\end{figure}

\clearpage
\appendix

\clearpage
\section*{Appendix A}\label{code:asdf}

Matlab code for asdf.

%\lstinputlisting{asdf.m}

\end{document}
